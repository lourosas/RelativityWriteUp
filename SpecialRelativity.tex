\documentclass[letterpaper]{article}
\usepackage{latexsym}
\usepackage{amsbsy}
\usepackage{amssymb}
\usepackage{amsmath}
\usepackage[makeroom]{cancel}
\usepackage{graphicx}
\title{A Synopsis of Special Relativity}
\begin{document}
\maketitle
\noindent
Relativity has been studied for the past one-hundred years.  This is
generic derivation of Special Relativity.\\\\
Special Relativity was developed by three prominent Physicsists and
Mathematicians.
\begin{enumerate}
\item Einstien
\item Lorentz
\item Poincar\'e
\end{enumerate}
Later, after originally published, Special Relativity was studied by
Minkowski.\\\\
This paper is a synopsis of the Mathematical Derivation of Special
Relativity.  The derivation is taking directly from Maxwell's
equations of Light, using the Gallilean Transform (Gallilean
Relativity).
\section{The Gallilean Transform}
The Gallilean Transform consists of the following equations:
\begin{align*}
x^\prime &= x-vt\\
y^\prime &= y\\
z^\prime &= z\\
t^\prime &= t
\end{align*}
And
\begin{align*}
x &= x^\prime+vt\\
y &= y^\prime\\
z &= z^\prime\\
t &= t^\prime
\end{align*}
Where the ``prime" frame is the frame of motion and the non-``prime"
frame is the ``stationary" frame.\\\\
As postulated by Einstein, there is \underline{no} ``stationary"
frame.  Hence, this idea is merely a basic understanding of the
difference between the observer in motion vs. the stationary observer.
For the purposes of this paper, the stationary
observer is the observer that ``sees" the other observer in motion.
\\\\
\section{Maxwell's Equations}
Simplistic version of Maxwell's equation is as follows:
\begin{equation*}
\nabla^2\varphi = \frac{1}{c^2}\frac{\partial^2\varphi}{\partial t^2}
\end{equation*}
\section{Gallilean Relativity into Maxwell's Equations}
Putting Gallilean Transforms into Maxwell's Equations, the equations
are the following:
\begin{align*}
\frac{\partial\varphi}{\partial x} &= \frac{\partial\varphi}{\partial x^\prime}\cancelto{1}{\frac{\partial x^\prime}{\partial x}}
+\cancelto{0}{\frac{\partial\varphi}{\partial y^\prime}\frac{\partial y^\prime}{\partial x}}
+\cancelto{0}{\frac{\partial\varphi}{\partial z^\prime}\frac{\partial z^\prime}{\partial x}}
+\cancelto{0}{\frac{\partial\varphi}{\partial t^\prime}\frac{\partial t^\prime}{\partial x}}
\\\therefore
\frac{\partial\varphi}{\partial x} &= \frac{\partial\varphi}{\partial x^\prime}
\end{align*}
And hence:
\begin{align*}
\frac{\partial^2\varphi}{\partial x^2} &= \frac{\partial^2\varphi}{\partial x^{\prime 2}}
\end{align*}
\begin{align*}
\frac{\partial\varphi}{\partial y} &= \cancelto{0}{\frac{\partial\varphi}{\partial x^\prime}\frac{\partial x^\prime}{\partial y}}
+\frac{\partial\varphi}{\partial y^\prime}\cancelto{1}{\frac{\partial y^\prime}{\partial y}}
+\cancelto{0}{\frac{\partial\varphi}{\partial z^\prime}\frac{\partial z^\prime}{\partial y}}
+\cancelto{0}{\frac{\partial\varphi}{\partial t^\prime}\frac{\partial t^\prime}{\partial y}}
\\\therefore
\frac{\partial\varphi}{\partial y} &= \frac{\partial\varphi}{\partial y^\prime}
\end{align*}
And hence:
\begin{align*}
\frac{\partial^2\varphi}{\partial y^2} &= \frac{\partial^2\varphi}{\partial y^{\prime 2}}
\end{align*}
\begin{align*}
\frac{\partial\varphi}{\partial z} &= \cancelto{0}{\frac{\partial\varphi}{\partial x^\prime}\frac{\partial x^\prime}{\partial z}}
+\cancelto{0}{\frac{\partial\varphi}{\partial y^\prime}\frac{\partial y^\prime}{\partial z}}
+\frac{\partial\varphi}{\partial z^\prime}\cancelto{1}{\frac{\partial z^\prime}{\partial z}}
+\cancelto{0}{\frac{\partial\varphi}{\partial t^\prime}\frac{\partial t^\prime}{\partial z}}
\\\therefore
\frac{\partial\varphi}{\partial z} &= \frac{\partial\varphi}{\partial z^\prime}
\end{align*}
And hence:
\begin{align*}
\frac{\partial^2\varphi}{\partial z^2} &= \frac{\partial^2\varphi}{\partial z^{\prime 2}}
\end{align*}
\begin{align*}
\frac{\partial\varphi}{\partial t} &= \frac{\partial\varphi}{\partial x^\prime}\cancelto{-v}{\frac{\partial x^\prime}{\partial t}}
+\cancelto{0}{\frac{\partial\varphi}{\partial y^\prime}\frac{\partial y^\prime}{\partial t}}
+\cancelto{0}{\frac{\partial\varphi}{\partial z^\prime}\frac{\partial z^\prime}{\partial t}}
+\frac{\partial\varphi}{\partial t^\prime}\cancelto{1}{\frac{\partial t^\prime}{\partial t}}
\\\therefore
\frac{\partial\varphi}{\partial t} &= \frac{\partial\varphi}{\partial t^\prime}-v\frac{\partial\varphi}{\partial x^\prime}
\bigg\}\alpha
\end{align*}
Thus:
\begin{align*}
\frac{\partial\varphi^2}{\partial t^2} &= \frac{\partial [\alpha]}{\partial t} =
\frac{\partial[\alpha]}{\partial x^\prime}\frac{\partial x^\prime}{\partial t} +
\frac{\partial[\alpha]}{\partial t^\prime}\frac{\partial t^\prime}{\partial t}
\\
\frac{\partial\varphi^2}{\partial t^2} &=
\frac{\partial[\alpha]}{\partial x^\prime}\cancelto{-v}{\frac{\partial x^\prime}{\partial t}} +
\frac{\partial[\alpha]}{\partial t^\prime}\cancelto{1}{\frac{\partial t^\prime}{\partial t}}
\\
\frac{\partial\varphi^2}{\partial t^2} &=
\frac{\partial[\alpha]}{\partial t^\prime}-
v\frac{\partial[\alpha]}{\partial x^\prime}
\\
\frac{\partial\varphi^2}{\partial t^2} &=
\frac{\partial}{\partial t^\prime}[\frac{\partial\varphi}{\partial t^\prime}-v\frac{\partial\varphi}{\partial x^\prime}]
-v\frac{\partial}{\partial x^\prime}[\frac{\partial\varphi}{\partial t^\prime}-v\frac{\partial\varphi}{\partial x^\prime}]
\\
\frac{\partial^2\varphi}{\partial t^2} &=
\frac{\partial^2\varphi}{\partial t^{\prime 2}}
-2v\frac{\partial\varphi^2}{\partial x \partial t^\prime}
+v^2\frac{\partial^2\varphi}{\partial x^{\prime 2}}
\\
\frac{1}{c^2}\frac{\partial^2\varphi}{\partial t^2} &=
\frac{1}{c^2}\frac{\partial^2\varphi}{\partial t^{\prime 2}}
-\frac{2v}{c^2}\frac{\partial\varphi^2}{\partial x \partial t^\prime}
+\frac{v^2}{c^2}\frac{\partial^2\varphi}{\partial x^{\prime 2}}
\end{align*}
Which follows:
\begin{align*}
\frac{\partial^2\varphi}{\partial x^{\prime 2}} +
\frac{\partial^2\varphi}{\partial y^{\prime 2}} +
\frac{\partial^2\varphi}{\partial z^{\prime 2}} &=
\frac{1}{c^2}\frac{\partial^2\varphi}{\partial t^{\prime 2}}
-\frac{2v}{c^2}\frac{\partial\varphi^2}{\partial x \partial t^\prime}
+\frac{v^2}{c^2}\frac{\partial^2\varphi}{\partial x^{\prime 2}}
\\\therefore
(1-\frac{v^2}{c^2})\frac{\partial^2\varphi}{\partial x^{\prime 2}} +
\frac{\partial^2\varphi}{\partial y^{\prime 2}} +
\frac{\partial^2\varphi}{\partial z^{\prime 2}} +
\frac{2v}{c^2}\frac{\partial\varphi^2}{\partial x \partial t^\prime}
&= \frac{1}{c^2}\frac{\partial^2\varphi}{\partial t^{\prime 2}}
\end{align*}
$
\boxed{
(1-\frac{v^2}{c^2})\frac{\partial^2\varphi}{\partial x^{\prime 2}} +
\frac{\partial^2\varphi}{\partial y^{\prime 2}} +
\frac{\partial^2\varphi}{\partial z^{\prime 2}} +
\frac{2v}{c^2}\frac{\partial\varphi^2}{\partial x \partial t^\prime}
= \frac{1}{c^2}\frac{\partial^2\varphi}{\partial t^{\prime 2}}}
$
\\
\begin{center}
\fbox{\parbox{200pt}{
This shows that plugging in Galilean Relativity into
Maxwell's Equations \textbf{is not} invariant.  This shows a problem
with the equations as related to Galilean Relativity.}}
\end{center}
Under the given situation of non-invariance, the question arises what
would the relativistic equations be to make Maxwell's equations
invariant?\\\\
From a mathematical point of view, the answer lies in the expression:
$
\boxed{
\frac{2v}{c^2}\frac{\partial\varphi^2}{\partial x \partial t^\prime}}
$
Which would imply a direct \textbf{relation to time and space}.
\section{A Correction to Galilean Relatity}
Based on the section above, a new set of Relativity equations need to
be developed in such a way as to ensure invariance between reference
frames.
\begin{align*}
x^\prime &= Ax + Bvt\\
y^\prime &= y\\
z^\prime &= z\\
t^\prime &= Dt + Ex
\end{align*}
Letting $x^\prime = 0$, $x = vt$, and $-Avt = Bvt$\\
$\therefore B = -A$ and
\begin{align*}
x^\prime &= A(x - vt)
\end{align*}
with:
\begin{align*}
t^\prime &= Dt + Ex
\end{align*}
These equations can be substituted into Maxwell's equations in the
similar way as previous.
\begin{align*}
\frac{\partial\varphi}{\partial x} &=
\frac{\partial\varphi}{\partial x^\prime}\cancelto{A}{\frac{\partial x^\prime}{\partial x}}
+\cancelto{0}{\frac{\partial\varphi}{\partial y^\prime}\frac{\partial y^\prime}{\partial x}}
+\cancelto{0}{\frac{\partial\varphi}{\partial z^\prime}\frac{\partial z^\prime}{\partial x}}
+\frac{\partial\varphi}{\partial t^\prime}\cancelto{E}{\frac{\partial t^\prime}{\partial x}}
\\\therefore
\frac{\partial\varphi}{\partial x} &= A\frac{\partial\varphi}{\partial x^\prime} +
E\frac{\partial\varphi}{\partial t^\prime}\bigg\}\alpha
\end{align*}
Since $y^\prime = y$ and $z^\prime = z$, there is no change to these
original equations.
\begin{align*}
\frac{\partial\varphi}{\partial t} &=
\frac{\partial\varphi}{\partial x^\prime}\cancelto{-Av}{\frac{\partial x^\prime}{\partial t}}
+\cancelto{0}{\frac{\partial\varphi}{\partial y^\prime}\frac{\partial y^\prime}{\partial t}}
+\cancelto{0}{\frac{\partial\varphi}{\partial z^\prime}\frac{\partial z^\prime}{\partial t}}
+\frac{\partial\varphi}{\partial t^\prime}\cancelto{E}{\frac{\partial t^\prime}{\partial t}}
\\\therefore
\frac{\partial\varphi}{\partial t} &=
E\frac{\partial\varphi}{\partial t^\prime} -
Av\frac{\partial\varphi}{\partial x^\prime}\bigg\}\beta
\end{align*}
Which leads to:
\end{document}
