%\documentclass[letterpaper]{article}
\documentclass[journal]{IEEEtran}
\usepackage{latexsym}
\usepackage{amsbsy}
\usepackage{graphicx}
\usepackage{enumerate}
\usepackage{amsmath}
\usepackage[makeroom]{cancel}
\begin{document}
\title{A Derivation of Special Relativity}
\author{Lou Rosas}
\maketitle
\noindent
\begin{abstract}
Special Relativity, while giving credit mainly to Albert Einstein, 
was the work of additional mathematicians and Scientists including:  
Hendrik Lorentz, Henri Poincar\'e, and Herman Minkowski.  While all of
these notable physicists and mathematicians are credited for work (it
is the work of Lorentz who originally developed the transforms),
Einstein is given credit for actual ``concept" of Special Relativity.
This paper is a summation of my personal studies (for the past twenty
eight years) in the subject.
\end{abstract}
\section{A Short History}
Special Relativity starts in part with the work of Maxwell's Equations
of Light.  In those equations, Maxwell shows that
\begin{enumerate}
\item Light is an electromagnetic wave
\item The Speed of light is a constant based on the wave equation
\begin{enumerate}
\item This was originally predicted in the general wave equation
\item From the equation we see the speed of light is:
$c = \frac{1}{\sqrt{\epsilon_0\mu_0}}$ where $\epsilon_0$ is the
permitivity of free space and $\mu_0$ is the permeability of free space
\end{enumerate}
\end{enumerate}
Hertz would later confirm what Maxwell theorized mathematically.
There were still several issues related to the work of Maxwell.  Of
particular interrest in this subject were the
\begin{enumerate}
\item Prediction and verification of the Lumiferous Aether--the
theoretical non-viscous medium that allows light (in the form of
waves)to travel through
free space
\item The Absolute Reference Frame--required to hold that time is
absolute for every observer regardless of referece frame as predicted
by Newton
\end{enumerate}
\subsection{The Search for the Speed of Light}
The search for the Speed of Light proceded in the late 19th-century.
In 1887, an experiment was done by Albert Michelson and Edward
Morely to determine several factors related to light this included:
\begin{enumerate}
\item The verification of the existence the aether
\item The first determination of the speed of light: $c$
\begin{enumerate}
\item From this, the values of $\epsilon_0$ and $\mu_0$ could be
better acertained
\end{enumerate}
\end{enumerate}
\subsubsection{The Results for the Michelson-Morley Experiment}
The Michelson-Morely experiment showed no evidence of the aether.
\\\\In
addition, the Michelson-Morely experiment showeed the speed of light
measurement did not vary based on the speed \emph{relative} to the
observer measuring the speed of light.
\paragraph{What does that mean?}  As stated above the speed of light
measured by the Michelson-Morely experiment was a constant and
indifferent to the reference frame of the observer.  Essentially, that
means that if an observer is going $.5c$ towards a light source, they
will \emph{not} measure the speed of light from that light source to
be $1.5c$, and if an
observer is going $.5c$ away from a light source, they will \emph{not}
measure the speed of light from that light source to be $.5c$; but
rather $c$ in \emph{both}
cases.
\paragraph{Questions Answered but Left ``Unanswered}By nature, the
wave equation \emph{shows} the wave's velocity is constant.  For
light, Maxwell's equation \underline{directly} verifies the speed of
light is a constant (see above), and that velocity of light is
\emph{independent of reference frame}.  Yet, the Michelson-Morely
experiment still had scientists confused and bewildered.  The main
reason is due to the lack of the verifcation of the aether.  Maxwell's
Equations were believed to work (or be valid) based on the aether.
Afterwards, many scientists proposed the aether to be a non-viscous
fluid:  that was not verified, regardless. As a
result, while there was empirical evidence of the speed of light,
one question was answered, this experiment created more questions.
\subsection{The History of the Subject as Presented}
As stated in the abstract, there were several scientists and
mathematicians involved in the study and development of Special
Relativity:  Einstein, Poincar\'e, Lorentz and (later) Minkowski.
While all worked on the theory of Special Relativity, Einstein is
given credit for the actual theory.  This is due to the fact that
Einstein presented not only the mathematics of the theory (that was
completed by Lorentz and Poincar\'e; Einstein working
independently ``discovered" the same equations); rather, Einstein
presented a physical description of the Special Theory of Relativity.
Mainly, Einstein presented three postulates to the theory that neither
Lorentz nor Poincar\'e presented:
\begin{enumerate}
\item The speed of light is a constant regardless of reference frame
\item There is no aether
\item (As a result from the above postulate) There is no absolute
reference frame
\end{enumerate}
\subsection{The Mathematics Presented}
The Mathematics derived in this paper are based on the work of
several of the scientists and mathematicians that worked on the theory
of Special Relativity:  Mainly Lorentz and Minkowski, as well as
Einstein.  When credit is known, credit will be given as appropriate.
\subsection{More History to follow}
It should be noted even though there is a ``history" section, this is
not meant to imply there will be no more history will follow in the
process of the derivation.
\section{Galillean Relativity}
This paper starts with Galillean Relativity.\\
The following are the equations commonly known as Galilean
Relativity:\\\\
\begin{eqnarray}
x^\prime = x-vt\\
y^\prime = y\\
z^\prime = z\\
t^\prime = t
\end{eqnarray}
and
\begin{eqnarray}
x = x^\prime+vt\\
y = y^\prime\\
z = z^\prime\\
t = t^\prime
\end{eqnarray}
Where the prime values (prime frame) are the position values for the
observer in the moving frame.\\
In the history of the study of Special Relativity, plugging the
Galillean Transforms (eqns 1-8) into Maxwell's Equations presented an
odd anomaly that showed the transforms were non-invariant.  From a
mathematical perspective, this seems to be the ``best place" to start
the derivation presented in this paper.
\subsection{Plugging Galillean Relativity into Maxwell's Equations}
\subsubsection{Maxwell's Equations}
From Maxwell, it was mathematically shown that light is an
electromagnetic wave:
\begin{eqnarray}
\nabla^2E=\mu_0\epsilon_0\ddot{E}\\
\nabla^2B=\mu_0\epsilon_0\ddot{B}
\end{eqnarray}
This can be written in the ``Generic Form" of the Wave Equation:
\begin{equation}
\nabla^2\varphi=\frac{1}{c^2}\frac{\partial^2 \varphi}{\partial t^2}
\end{equation}
Where
\begin{equation}
c = \frac{1}{\sqrt{\epsilon_0\mu_0}}
\end{equation}
and $\epsilon_0$ is the
permitivity of free space and $\mu_0$ is the permeability of free
space.
\subsubsection{Putting Galillean Relativity Equations in Maxwell's
equations}
Using the equations:
\begin{eqnarray}
x^\prime = x-vt\\
y^\prime = y\\
z^\prime = z\\
t^\prime = t
\end{eqnarray}
Given:
\begin{equation}
\nabla^2 = \frac{\partial^2}{\partial x^2} +  
                   \frac{\partial^2}{\partial y^2} +
                   \frac{\partial^2}{\partial z^2}
\end{equation}
From the chain rule of partial derivatives, it can be shown that:
\begin{equation}
\frac{\partial\varphi}{\partial x}=
\frac{\partial\varphi}{\partial x^\prime}
\cancelto{1}{\frac{\partial x^\prime}{\partial x}} +
\cancelto{0}{\frac{\partial\varphi}{\partial y^\prime}
\frac{\partial y^\prime}{\partial x}} +
\cancelto{0}{\frac{\partial\varphi}{\partial z^\prime}
\frac{\partial z^\prime}{\partial x}} +
\cancelto{0}{\frac{\partial\varphi}{\partial t^\prime}
\frac{\partial t^\prime}{\partial x}}
\end{equation}
Therefore:
\begin{equation}
\boxed{
\frac{\partial\varphi}{\partial x} = 
\frac{\partial\varphi}{\partial x^\prime}}
\end{equation}
\begin{equation}
\boxed{
Let \quad \alpha\equiv\frac{\partial\varphi}{\partial x^\prime}}
\end{equation}
Continuing, the same thing can be done for $y, z$ and $t$.
\begin{equation}
\frac{\partial\varphi}{\partial y}=
\cancelto{0}{\frac{\partial\varphi}{\partial x^\prime}
\frac{\partial x^\prime}{\partial y}} +
\frac{\partial\varphi}{\partial y^\prime}
\cancelto{1}{\frac{\partial y^\prime}{\partial y}} +
\cancelto{0}{\frac{\partial\varphi}{\partial z^\prime}
\frac{\partial z^\prime}{\partial y}} +
\cancelto{0}{\frac{\partial\varphi}{\partial t^\prime}
\frac{\partial t^\prime}{\partial y}}
\end{equation}
Therefore:
\begin{equation}
\boxed{
\frac{\partial\varphi}{\partial y} = 
\frac{\partial\varphi}{\partial y^\prime}}
\end{equation}
\begin{equation}
\boxed{
Let \quad \beta\equiv\frac{\partial\varphi}{\partial y^\prime}}
\end{equation}
\begin{equation}
\frac{\partial\varphi}{\partial z}=
\cancelto{0}{\frac{\partial\varphi}{\partial x^\prime}
\frac{\partial x^\prime}{\partial z}} +
\cancelto{0}{\frac{\partial\varphi}{\partial y^\prime}
\frac{\partial y^\prime}{\partial z}} +
\frac{\partial\varphi}{\partial z^\prime}
\cancelto{1}{\frac{\partial z^\prime}{\partial z}} +
\cancelto{0}{\frac{\partial\varphi}{\partial t^\prime}
\frac{\partial t^\prime}{\partial z}}
\end{equation}
Therefore:
\begin{equation}
\boxed{
\frac{\partial\varphi}{\partial z} = 
\frac{\partial\varphi}{\partial z^\prime}}
\end{equation}
\begin{equation}
\boxed{
Let \quad \gamma\equiv\frac{\partial\varphi}{\partial z^\prime}}
\end{equation}
The partial with respect to $t$ is more tricky:
\begin{equation}
\frac{\partial\varphi}{\partial t}=
\frac{\partial\varphi}{\partial x^\prime}
\cancelto{-v}{\frac{\partial x^\prime}{\partial t}} +
\cancelto{0}{\frac{\partial\varphi}{\partial y^\prime}
\frac{\partial y^\prime}{\partial t}} +
\cancelto{0}{\frac{\partial\varphi}{\partial z^\prime}
\frac{\partial z^\prime}{\partial t}} +
\frac{\partial\varphi}{\partial t^\prime}
\cancelto{1}{\frac{\partial t^\prime}{\partial t}}
\end{equation}
\begin{equation}
\boxed{
\frac{\partial\varphi}{\partial t} = 
\frac{\partial\varphi}{\partial t^\prime} - v\frac{\partial\varphi}
{\partial x^\prime}}
\end{equation}
\begin{equation}
\boxed{
Let \quad \Delta\equiv\frac{\partial\varphi}{\partial t^\prime} -
v\frac{\partial\varphi}{\partial x^\prime}}
\end{equation}
Using \underline{ $\alpha$} substitution from above:
\begin{equation}
\frac{\partial^2\varphi}{\partial x^2} = \frac{\partial}{\partial x}
\left[\frac{\partial\varphi}{\partial x}\right] =
\frac{\partial}{\partial x}
\left[\frac{\partial\varphi}{\partial x^\prime}\right] =
\frac{\partial\alpha}{\partial x}
\end{equation}
\begin{equation}
\frac{\partial\alpha}{\partial x} =
\frac{\partial\alpha}{\partial x^\prime}
\cancelto{1}{\frac{\partial x^\prime}{\partial x}} +
\cancelto{0}{\frac{\partial\alpha}{\partial y^\prime}
\frac{\partial y^\prime}{\partial x}} +
\cancelto{0}{\frac{\partial\alpha}{\partial z^\prime}
\frac{\partial z^\prime}{\partial x}} +
\cancelto{0}{\frac{\partial\alpha}{\partial t^\prime}
\frac{\partial t^\prime}{\partial x}}
\end{equation}
\begin{equation}
\frac{\partial\alpha}{\partial x} = \frac{\partial}{\partial x^\prime}
\left[\frac{\partial\varphi}{\partial x^\prime}\right] =
\frac{\partial^2\varphi}{\partial x^{\prime 2}}
\end{equation}
Therefore:
\begin{equation}
\boxed{
\frac{\partial^2\varphi}{\partial x^2} = 
\frac{\partial^2\varphi}{\partial x^{\prime 2}}}
\end{equation}
Using the \underline{$\beta$} substitution from above:
\begin{equation}
\frac{\partial^2\varphi}{\partial y^2} = \frac{\partial}{\partial y}
\left[\frac{\partial\varphi}{\partial y}\right] =
\frac{\partial}{\partial y}
\left[\frac{\partial\varphi}{\partial y^\prime}\right] =
\frac{\partial\beta}{\partial y}
\end{equation}
\begin{equation}
\frac{\partial\beta}{\partial y} =
\cancelto{0}{\frac{\partial\beta}{\partial x^\prime}
\frac{\partial x^\prime}{\partial y}} +
\frac{\partial\beta}{\partial y^\prime}
\cancelto{1}{\frac{\partial y^\prime}{\partial y}} +
\cancelto{0}{\frac{\partial\beta}{\partial z^\prime}
\frac{\partial z^\prime}{\partial y}} +
\cancelto{0}{\frac{\partial\beta}{\partial t^\prime}
\frac{\partial t^\prime}{\partial y}}
\end{equation}
\begin{equation}
\frac{\partial\beta}{\partial y} = \frac{\partial}{\partial y^\prime}
\left[\frac{\partial\varphi}{\partial y^\prime}\right] =
\frac{\partial^2\varphi}{\partial y^{\prime 2}}
\end{equation}
Therefore:
\begin{equation}
\boxed{
\frac{\partial^2\varphi}{\partial y^2} = 
\frac{\partial^2\varphi}{\partial y^{\prime 2}}}
\end{equation}
By the similarites above (using the \underline{$\gamma$}
substitution):
\begin{equation}
\boxed{
\frac{\partial^2\varphi}{\partial z^2} = 
\frac{\partial^2\varphi}{\partial z^{\prime 2}}}
\end{equation}
Using $\Delta$ substitution:
\begin{equation}
\frac{\partial^2\varphi}{\partial t^2} = \frac{\partial}{\partial t}
\left[\frac{\partial\varphi}{\partial t}\right]=
\frac{\partial}{\partial t}
\left[\frac{\partial\varphi}{\partial t^\prime} - 
v\frac{\partial\varphi}{\partial x^\prime}
\right] = \frac{\partial\Delta}{\partial t}
\end{equation}
\begin{equation}
\frac{\partial\Delta}{\partial t} =
\frac{\partial\Delta}{\partial x^\prime}
\cancelto{-v}{\frac{\partial x^\prime}{\partial t}} +
\cancelto{0}{\frac{\partial\Delta}{\partial y^\prime}
\frac{\partial y^\prime}{\partial t}} +
\cancelto{0}{\frac{\partial\Delta}{\partial z^\prime}
\frac{\partial z^\prime}{\partial t}} +
\frac{\partial\Delta}{\partial t^\prime}
\cancelto{1}{\frac{\partial t^\prime}{\partial t}}
\end{equation}
\begin{equation}
\frac{\partial\Delta}{\partial t} =
\frac{\partial\Delta}{\partial t^\prime}
- v \frac{\partial\Delta}{\partial x^\prime}
\end{equation}
\begin{equation}
\frac{\partial\Delta}{\partial t} = 
\frac{\partial}{\partial t^\prime}
\left[\frac{\partial\varphi}{\partial t^\prime} -
v\frac{\partial\varphi}{\partial x^\prime}\right] - v
\frac{\partial}{\partial x^\prime}
\left[\frac{\partial\varphi}{\partial t^\prime} -
v\frac{\partial\varphi}{\partial x^\prime}\right]
\end{equation}
\begin{equation}
\frac{\partial\Delta}{\partial t} = 
v^2\frac{\partial^2\varphi}{\partial x^{\prime2}} - 2v
\frac{\partial^2\varphi}{\partial x^\prime\partial t^\prime} +
\frac{\partial^2\varphi}{\partial t^{\prime2}}
\end{equation}
Therefore:
\begin{equation}
\frac{\partial^2\varphi}{\partial t^2} =
v^2\frac{\partial^2\varphi}{\partial x^{\prime2}} - 2v
\frac{\partial^2\varphi}{\partial x^\prime\partial t^\prime} +
\frac{\partial^2\varphi}{\partial t^{\prime2}}
\end{equation}
and:
\begin{equation}
\boxed{
\frac{1}{c^2}\frac{\partial^2\varphi}{\partial t^2} =
\frac{v^2}{c^2}\frac{\partial^2\varphi}{\partial x^{\prime2}} -
\frac{2v}{c^2}
\frac{\partial^2\varphi}{\partial x^\prime\partial t^\prime} +
\frac{1}{c^2}\frac{\partial^2\varphi}{\partial t^{\prime2}}
}
\end{equation}
Combining equations 33,37,38 and 45, equation 11 becomes:
\begin{equation}
\boxed{
\frac{\partial^2\varphi}{\partial x^{\prime2}} +
\frac{\partial^2\varphi}{\partial y^{\prime2}} +
\frac{\partial^2\varphi}{\partial z^{\prime2}} =
\frac{v^2}{c^2}\frac{\partial^2\varphi}{\partial x^{\prime2}} -
\frac{2v}{c^2}
\frac{\partial^2\varphi}{\partial x^\prime\partial t^\prime} +
\frac{1}{c^2}\frac{\partial^2\varphi}{\partial t^{\prime2}}
}
\end{equation}
or:
\begin{equation}
\boxed{
\left(1 - \frac{v^2}{c^2}\right)
\frac{\partial^2\varphi}{\partial x^{\prime2}} +
\frac{\partial^2\varphi}{\partial y^{\prime2}} +
\frac{\partial^2\varphi}{\partial z^{\prime2}}  +
\frac{2v}{c^2}
\frac{\partial^2\varphi}{\partial x^\prime\partial t^\prime} =
\frac{1}{c^2}\frac{\partial^2\varphi}{\partial t^{\prime2}}
}
\end{equation}
This violates the principle of invariance.  Another way of putting
this:  the equations are non-invariant.\\\\The demonstration of
non-invariace continues with the related derivation:
\begin{equation}
\frac{\partial\varphi}{\partial x^{\prime}} =
\frac{\partial\varphi}{\partial x}\cancelto{1}
{\frac{\partial x}{\partial x^\prime}} +
\cancelto{0}{\frac{\partial\varphi}{\partial y}
\frac{\partial y}{\partial x^\prime}} +
\cancelto{0}{\frac{\partial\varphi}{\partial z}
\frac{\partial z}{\partial x^\prime}} +
\cancelto{0}{\frac{\partial\varphi}{\partial t}
\frac{\partial t}{\partial x^\prime}}
\end{equation}
\begin{equation}
\boxed{
Let\quad\alpha\equiv\frac{\partial\varphi}{\partial x} 
}
\end{equation}
Continuing with $y^\prime$,$z^\prime$ and $t^\prime$:\\
\begin{equation}
\frac{\partial\varphi}{\partial y^{\prime}} =
\cancelto{0}{\frac{\partial\varphi}{\partial x}\frac{\partial x}{\partial y^\prime}}
\end{equation}
\end{document}
