\documentclass[journal]{IEEEtran}
%\documentclass{article}
%\documentclass{report}
\usepackage{graphicx}
\usepackage{amsmath}

\begin{document}
\title{Basic Special Relativity Equations:  A Practice}
\author{Lou Rosas}
\maketitle

\begin{abstract}
Galilean Relativity as applied to Maxwell's equations does not work.
Presented is a simple derivation of Lorentz Equations that
essentially defined Special Relativity developed by Einstien and
Poincare'.  In addition, I also did this to practice my
\LaTeX.
\end{abstract}
\section{Galillean Relativity}
The following are the equations commonly known as Galilean
Relativity:\\\\
\begin{eqnarray}
x^\prime = x-vt\\
y^\prime = y\\
z^\prime = z\\
t^\prime = t
\end{eqnarray}
\subsection{Plugging Galillean Relativity into Maxwell's Equations}
\subsubsection{Maxwell's Equations}
From Maxwell, it was mathematically shown that light is an
electromagnetic wave:
\begin{eqnarray}
\nabla^2E=\mu_0\epsilon_0\ddot{E}\\
\nabla^2B=\mu_0\epsilon_0\ddot{B}
\end{eqnarray}
This can be written in the ``Generic Form" of the Wave Equation:
\begin{equation}
\nabla^2\varphi=\frac{1}{c^2}\frac{\partial^2 \varphi}{\partial t^2}
\end{equation}
\subsubsection{Putting Galillean Relativity Equations in Maxwell's
equations}

\end{document}